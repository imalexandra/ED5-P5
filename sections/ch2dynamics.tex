\section{Quadcopter Flight Dynamics}
Before we begin doing any work, it is important to understand how quadcopter works. A quadcopter, as the name suggest, runs on four motors usually placed at equal distances from each other (figure here) \newline
By controlling the speed at which the motors rotate, it is possible to change how quadcopter moves. Movement can be broken down in 3 separate sets of motor speeds:\newline
1) If the two back motors rotate faster than the two frontal motors, the quadcopter will pitch forward. Switching the speeds will result in aft pitch.
2) If the two left motors have higher RPM than the two motors on the right side, the vehicle will roll to the right. Swapping the speeds differences will result in a roll to the left.
3) Increasing the speed of one of the diagonal pair of the motors will result in yaw to the direction of the torque of the motors. The copter will then spin around its axis.\newline
Now that the effect of speed of the motors is established, it is important to talk about the forces in play. There are 4 main forces affecting quadcopter - thrust, lift, draw and drop (figure again)\newline
The drop - or the gravitational force - affects the vehicle at all times. As any object on Earth, its mass is driven towards the centre of the planet.\newline
The thrust is the force generated by the motors that is allowing the vehicle to move horizontally.\newline
The lift force is also generated by the motors and allows the vehicle to move up.\newline
Draw force - ???\newline
A powered off motors is only affected by the drop force and therefore stays on the ground. In order to lift it up, we need to understand the relationship between quadcopter and the thrust to weight ratio - or TWR for short. This ratio can be determined by equation Ft/Fg and describes the vehicle's ability to move up. With TWR expressed as a number, three cases can be identified:
1) TWR < 1: The gravitational force is higher than the lift force and therefore the quadcopter is drawn towards the ground.
2) TWR = 1: The forces are equal, causing quadcopter's altitude to stay constant.
3) TWR > 1: The thrust is higher than drop force, allowing vehicle to move upwards.\newline
Therefore, in order to get a quadcopter up in the air, it is necessary to generate enough thrust for TWR ration to be higher than one. In order to land it, the TWR must be smaller than 1, allowing the quadcopter to move downwards.\newline
