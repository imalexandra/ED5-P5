Quadcopter control is a complex, yet interesting problem. One of the reasons why this control problem is challenging is the fact that a quadcopter has six degrees of freedom, but only four inputs which affects the linearity of the dynamics and makes the quadcopter underactuated. One other important thing to mention is that quadcopters, unlike ground vehicles, have very little friction that prevents their motion, so they have to provide their own damping in order to be able to stop or maintain stability.

This chapter will give a general overview on the pieces of technical equipment which we are using and show what the purpose of each one is. It will also present how our scopes - hovering and maneuvering - can be achieved by designing and implementing a control system. A solution in terms of the structure of our control system will be identified. We will also delve into the basic working principle of a quadcopter/flight dynamics.

\section{Physical Setup}
\subsection{Motors}
Controlling a quadcopter can be done efficiently by using high-quality motors with fast response, which will ensure more of a stable flight. The motors must also be powerful enough to be able to lift the quadcopter and perform the required aerial movements. 

The motor that we are using is the xx Brushless DC Motor. -more on this later after looking at specs-

-insert figure

\subsection{Propellers}
The propellers don't have such strict requirements as the motors. They are needed to be light and have a size and lift potential in order for the quadcopter to hover at less than 50\% of the motor capacity. For our quadcopter, we are using plasting -insert size- propellers with light weight. The length is given by the first measurement - x, while the pitch by the second one - x.

-insert figure-

\subsection{Electric Speed Controller}
Electronic Speed Controller (ESC) is a widely used device in rotorcrafts. The purpose of an ESC is to vary the electric motor's speed. They also come with programmable features, such as braking or selecting appropriate type of battery. We need the ESC to have a fast response, for the same reasons mentioned for the motors in -section-. The ESC that we are using is the TURNIGY PLUSH-30A.

+more from datasheet

-insert figure-

\subsection{Ardupilot}

ardupilot - features: accelerometer, gyroscope, magnetometer 3 axis

\subsection{Power Distribution Board}
To reduce the number of connections straight to the battery, we used the xx Power distribution board from xx. A board like is an easy solution since all we have to do is connect the four ESCs to the board and then connect the board to the battery.

-insert figure-

\subsection{Battery}
specs.

-insert figure-

\section{Flight Dynamics}
Before we begin doing any work, it is important to understand how a quadcopter works. A quadcopter, as the name suggest, runs on four motors, usually placed at equal distances from each other. An example can be seen in figure \ref{droneIdle}.
\begin{figure}[H]
  \centering
    \includegraphics[width=0.7\textwidth]{images/droneIdle.png}
	\caption{A quadcopter viewed from top}
	\label{droneIdle}
\end{figure}

The direction of rotation and produced torque as well as the amount of force generated by the motors affect the quadcopter's movement. See figure \ref{droneDirections} for an example in an idle - or hovering - position.
\begin{figure}[H]
  \centering
    \includegraphics[width=0.7\textwidth]{images/droneDirections.png}
	\caption{Motor direction overview}
	\label{droneDirections}
\end{figure}
Here, the yellow arrows indicate the direction of rotation, the blue arrows show the direction of torque and the green arrows display the amount of force generated by the motors.

By controlling the speed at which the motors rotate, it is possible to change how quadcopter moves. Movement can be broken down in 3 separate sets of motor speeds:

1) If the two back motors rotate faster than the two frontal motors, the quadcopter will pitch forward. Switching the speeds will result in aft pitch. The required change in speeds for a pitch forward can be seen in figure \ref{dronePitch}.
\begin{figure}[H]
  \centering
    \includegraphics[width=0.7\textwidth]{images/dronePitch.png}
	\caption{Motor RPM requirements for a pitch forward}
	\label{dronePitch}
\end{figure}
2) If the two left motors have higher RPM than the two motors on the right side, the vehicle will roll to the right. Swapping the speeds differences will result in a roll to the left. See figure \ref{droneRoll} for an example of a roll to the right.
\begin{figure}[H]
  \centering
    \includegraphics[width=0.7\textwidth]{images/droneRoll.png}
	\caption{RPM requirements for a roll to the right}
	\label{droneRoll}
\end{figure}
3) Increasing the speed of one of the diagonal pair of the motors will result in yaw to the direction of the torque of the motors. The copter will then spin around its axis. Motor speed changes causing a yaw to the right can be seen in figure \ref{droneYaw}.
\begin{figure}[H]
  \centering
    \includegraphics[width=0.7\textwidth]{images/droneYaw.png}
	\caption{The two diagonal motors with increased RPM cause yaw to the right}
	\label{droneYaw}
\end{figure}

(Re-visit this part)
Now that the effect of speed of the motors is established, it is important to talk about the forces in play. There are 4 main forces affecting quadcopter - thrust, lift, draw and drop (figure again)

The drop - or the gravitational force - affects the vehicle at all times. As any object on Earth, its mass is driven towards the centre of the planet.

The thrust is the force generated by the motors that is allowing the vehicle to move horizontally.

The lift force is also generated by the motors and allows the vehicle to move up.

Draw force - ???

A powered off motors is only affected by the drop force and therefore stays on the ground. In order to lift it up, we need to understand the relationship between quadcopter and the thrust to weight ratio - or TWR for short. This ratio can be determined by equation Ft/Fg and describes the vehicle's ability to move up. With TWR expressed as a number, three cases can be identified:
1) TWR < 1: The gravitational force is higher than the lift force and therefore the quadcopter is drawn towards the ground.
2) TWR = 1: The forces are equal, causing quadcopter's altitude to stay constant.
3) TWR > 1: The thrust is higher than drop force, allowing vehicle to move upwards.

Therefore, in order to get a quadcopter up in the air, it is necessary to generate enough thrust for TWR ration to be higher than one. In order to land it, the TWR must be smaller than 1, allowing the quadcopter to move downwards.



