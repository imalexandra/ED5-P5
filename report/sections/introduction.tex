\section{Introduction}

The theme of this semester's project lies within \textit{Automation}. Automation can be simply described as being the use of diverse control systems for fulfilling a certain task with little to no human interaction. As known from the previous semesters, a control system is an instrument which has the role of adapting the behaviour of a system according to a desired state, also knows as steady-steady or reference. Any control system has three components: measurement, control and actuation. Without one of these, automation would not be possible. Essentially, the measure reflects the current state of the system, the controller is the brain that given the measurement decides which action will be performed and the actuator is the one executing the action. 

Project ideas around the topic of automation are unlimited, since it is so widely spread. Having discussed a few of them that would meet the semester's requirements, we finally decided to work on the control of a quadcopter. Our decision was, for the most part, based on the fact that the university had the required equipment available, which enabled us to start working on the project right away. 

Unmanned Aerial Vehicles (UAVs) have been attracting attention for many decades now. Powered UAVs were, at first, utilized by the military to execute reconnaissance missions. Nowadays, they have found other uses, such as aerial photography, search and rescue, delivery, geographic mapping and more. While there are different types of UAVs, our focus is on the multirotors. Multirotor can be defined as a rotorcraft with more than two motors. Based on this, the four most common types are tricopter, quadcopter, hexecopter and octocopter, each having 3, 4, 6 and 8 motors respectively. Each type of multirotor has its ups and downs - more motors mean higher liftforce and more reliable stability, but they also increase both price and damage caused in case of an error. Due to their price, size and ease of setup, quadcopters are the most popular type. They are popularly referred to as drones. Our goal for the project is to design a control system that makes it possible for the quadcopter to be stable - hovering mid-air. Basically, our input will be a certain height and the quadcopter will have to automatically adjust to that height and maintain its stability when no further inputs are given. 
