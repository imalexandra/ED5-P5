\section{Introduction}

The theme of this semester's project lies within \textit{Automation}. Automation can be simply described as being the use of diverse control systems for fulfilling a certain task with little to no human interaction. As known from the previous semester, a control system is an instrument which has the role of adapting the behaviour of a system according to a desired state, also knows as steady-steady or reference. Any control system has three components: measurement, control and actuation. Without one of these, automation would not be possible. Essentially, the measure reflects the current state of the system, the controller is the brain that given the measurement decides which action will be performed and the actuator is the one executing the action. 

Project ideas around the topic of automation are unlimited, since it is so widely spread. Having discussed a few of them that would meet the semester's requirements, we finally decided to work on the control of a quadcopter. Our decision was for the most part based on the fact that the university had the required equipment available, which enabled us to start working on the project right-away. 

Quadcopters are popularly referred to as drones and they have increased their areas of operation from the military sector to more commercial uses such as search and rescue/healthcare, geographic mapping, aerial photography, surveillance etc. Our goal for the project is to design a control system that makes it possible for the quadcopter to be stable - hovering and to also act according to the user's input - manouvering. Basically, our input will be a certain height and the quadcopter has to automatically adjust to that height and mentain its' stability when no further inputs are given. A safety feature - obstacle aviodance will also be implemented. 
