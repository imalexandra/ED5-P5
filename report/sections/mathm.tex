\section{Motor Dynamics}
In this chapter we will take a look at mathematical models that describe our quadcopter system.
To begin with, we can identify an equation that describes torque generated by a motor:
\begin{equation}
\label{torque1}
	Q = K_qI
\end{equation}
Here, \textit{Q} stands for torque, \textit{I} - current and \textit{Kq} - motor constant, relating current to the torque.
Another equation can be identified as:
\begin{equation}
\label{voltage1}
	V = R_aI + K_e\omega
\end{equation}
where \textit{V} is the voltage, \textit{$R_a$} is the motors armature resistance, \textit{$K_e$} is the back EMF constant and \textit{$\omega$} is motor's angular rate.

We can then convert voltage into power in a steady state to get the following equation:
\begin{equation}
\label{power1}
	P = IV = \frac{Q}{K_q}V
\end{equation}
\textit{P} - power, which can be related to thrust by equating the power produced by the motors to the ideal power required to generate thrust by increasing the momentum of a column of air. This ideal power \textit{$P_h$}, when hovering, can be found using the following equation:
\begin{equation}
\label{power2}
	P_h = T\upsilon_h
\end{equation}
Here, \textit{T} - thrust force and \textit{$\upsilon_h$} is the induced velocity when hovering. This velocity is the change in air speed which is induced by the motor blades with respect to the free stream velocity. However, to simplify the model and to reflect our testing conditions, this free stream velocity is set to zero due to lack of wind force.
Using momentum theory, we can identify another equation:
\begin{equation}
\label{velocity1}
	\upsilon_h = \sqrt{\frac{T}{2\rho A}}
\end{equation}
where \textit{$\rho$} - air density and \textit{A} is the area covered by the blade. This area is equal to $\pi$ multiplied by $R^2$, which is the radius of the blade.

Since the torque is proportional to the thrust force generated by the motor with a constant ratio \textit{$K_t$}, which depends on blade geometry, we can find the relation between the applied voltage and the thrust by combining equations \ref{power2} and \ref{velocity1}:
\begin{equation}
\label{voltage2}
	\frac{Q}{K_q}V = \frac{K_tT}{K_q}V = \frac{T^\frac{3}{2}}{\sqrt{2\rho A}}
\end{equation}
We then get the following equation:
\begin{equation}
\label{thrust1}
	T = \frac{2\rho AK_t^2}{K_q^2}V^2
\end{equation}

This final equation describes the relationship between the constants, applied voltage and the torque produced by the motors.