\pdfbookmark[0]{English title page}{label:titlepage_en}
\aautitlepage{%
  \englishprojectinfo{
    Hovering control of a quadcopter
  }{%
    Scientific Theme %theme
  }{%
    Autumn Semester 2016 %project period
  }{%
     ED5-3-E16
  }{%
    %list of group members
	Alexandra Dorina Török\\	
	Andrius Kulšinskas\\
  }{%
    %list of supervisors
	Christian Mai\\
	Zhenyu Yang\\
  }{%
    3 % number of printed copies
  }{%
    \today % date of completion
  }%
}{%department and address
  \textbf{School of Information and Communication Technology}\\
  Niels Bohrs Vej 8\\
  DK-6700 Esbjerg\\
  \href{http://sict.aau.dk}{http://sict.aau.dk}
}{
	This report describes the implementation of a control system using state-space representation of a quadcopter. The end goal for the control system is to stabilize both attitude and altitude of the quadcopter, therefore making the hovering and maneuvering movements possible. The system is a Multiple Input - Multiple Output system and it uses mathematical modeling together with experiments to make the creation of a system model possible and, therefore enabling us to design a controller that would stabilize the outputs. The system is made out of four BLDC motors, four ESCs, four propellers, four infrared sensors, an Ardupilot board and a  quadcopter frame. The motors and propellers are the actuators of the system, and the accelerometer and gyroscope sensors are the sensors. The report also delves into filter design and system response in both open-loop and closed loop scenarios.  %ABSTRACT
}