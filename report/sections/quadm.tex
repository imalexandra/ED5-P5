Quadcopter control is a complex, yet interesting problem. One of the reasons why this control problem is challenging is the fact that a quadcopter has six degrees of freedom, but only four inputs which affects the linearity of the dynamics and makes the quadcopter underactuated. 

In this chapter we will take a look at the kinematics and dynamics equations that describe our quadcopter system.

\section{Quaternions and Euler angles}
In Section \label{2.1}, we have explained how the position of the quadcopter is expressed in the inertial frame and how the  velocity of the quadcopter is express in the body-fixed frame. Therefore, we need to be able to find a relationship between the two, so that we can move from one to the other.

Euler angles have to be applied as a sequence of rotations. This report will use the \textit{roll, pitch, yaw} convention. Therefore, the roll rotation will be $(R(\phi)^{T})$, the pitch rotation will be $(R(\theta)^{T})$ and the yaw rotation will be $(R(\psi)^{T})$. Equations \label{rollrotation}, \label{pitchrotation} and \label{yawrotation}  describe the quadcopter's orientation relative to the inertial frame in matrix form:

\begin{equation}
\label{rollrotation}	
 	R(\phi)^{T}=\begin{bmatrix}
 	1 & 0 & 0 \\
 	0 & cos\phi & sin\phi \\
 	0 & -sin\phi & cos\phi \\
 	\end{bmatrix}
 \end{equation}
 
 \begin{equation}
\label{pitchrotation}	
 	R(\theta)^{T}=\begin{bmatrix}
 	cos\theta & 0 & -sin\theta \\
 	0 & 1 & 0 \\
 	sin\theta & 0 & cos\theta \\
 	\end{bmatrix}
 \end{equation}
 
 \begin{equation}
\label{yawrotation}	
 	R(\psi)^{T}=\begin{bmatrix}
 	cos\psi & sin\psi & 0 \\
 	-sin\psi & cos\psi & 0 \\
 	0 & 0 & 1 \\
 	\end{bmatrix}
 \end{equation}
 
Merging the three rotations as: 

 \begin{equation}
\label{S1}	
 	S=R(\phi)^{T}R(\theta)^{T}R(\psi)^{T}
 \end{equation}
 
gives the rotation matrix - \textit{S}, which expresses a vector from the inertial frame to the body-fixed frame: 
 
  \begin{equation}
\label{S2}
 S=\begin{bmatrix}
 	cos\theta cos\psi & cos\theta sin\psi & -sin\theta \\
 	cos\psi sin\phi sin\theta-cos\phi sin\psi & sin\psi sin\theta sin\phi+cos\phi cos\psi & cos\theta sin\phi \\
 	cos\psi cos\phi sin\theta & cos\phi sin\theta sin\psi-cos\psi sin\phi & cos\theta cos\phi \\
 	\end{bmatrix}
 	 \end{equation}
 	 
We can notice that if we were to have $\theta=\Pi/2$, the rotation matrix brings out a singularity by turning into:

  \begin{equation}
\label{S3}
 S=\begin{bmatrix}
 	0 & 0 & -1 \\
 	sin(\phi-\psi) & cos(\phi-\psi) & 0 \\
 	cos(\phi-\psi) & -sin(\phi-\psi) & 0 \\
 	\end{bmatrix}
 	 \end{equation}
 	 
As a result, one degree of freedom in the three dimensional space is lost. In addition, a change in either $\phi$ or $\psi$ will now have the same effect, which causes confusion. In order to be able to avoid this problem, a quaternion-based method can be applied instead. 

A quaternion can be expressed as $q=[q_{0}, q_{1}, q_{2}, q_{3}]^{T}$, which yields:

  \begin{equation}
\label{S4}
 q=\begin{bmatrix}
 	cos(\phi/2) cos(\theta/2) cos(\psi/2) + sin(\phi/2) sin(\theta/2) sin(\psi/2) \\
 	sin(\phi/2) cos(\theta/2) cos(\psi/2) - cos(\phi/2) sin(\theta/2) sin(\psi/2) \\
 	cos(\phi/2) sin(\theta/2) cos(\psi/2) + sin(\phi/2) cos(\theta/2) sin(\psi/2) \\
 	cos(\phi/2) cos(\theta/2) sin(\psi/2) - sin(\phi/2) sin(\theta/2) cos(\psi/2)
 	\end{bmatrix}
 	 \end{equation}
 	 
The quaternion and Euler angles are equivalent in terms of attitude, therefore we can convert from one representation to the other. The rotation matrix can be then rewritten as:

  \begin{equation}
\label{S5}
 S_{q}=\begin{bmatrix}
 	1-2(q_{2}^{2}+q_{3}^{2}) & 2(q_{1}q_{2}+q_{3}q_{0}) & 2(q_{1}q_{3}-q_{2}q_{0}) \\
 	2(q_{1}q_{2}-q_{3}q_{0}) & 1-2(q_{1}^{2}+q_{3}^{2}) & 2(q_{2}q_{3}+q_{1}q_{0})  \\
 	2(q_{1}q_{3}+q_{2}q_{0}) & 2(q_{2}q_{3}-q_{1}q_{0}) & 1-2(q_{1}^{2}+q_{2}^{2}) \\
 	\end{bmatrix}
 	 \end{equation}

\section{Quaternions Representation}

\section{Euler Representation}
 	 
 	 